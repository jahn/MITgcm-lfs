%%
%%  $Header$
%%  $Name$
%%

Package ``ICEFRONT'' is an extension of pkg/shelfice that deals with vertical
ice fronts for application to Greenland glaciers.  It includes

1) a routine (similar to dwnslp_init_fixed.F) which locates 3 things:
  a) horizontal locations of wet grid points adjacent to glacier calving faces
  b) vertical location of points in a)
  c) ratio of vertical face area to horizontal grid area for points in a)

2) a routine similar to shelfice_thermodynamics which applies the ice shelf
   melt/freeze equations to locations detected in 1) above and which scales
   them by 1c

3) a routine similar to shelfice_forcing which applies the forcing computed
   in 2) to locations computed in 1)

The package can be used in conjunction with pkg/shelfice, if both the
horizontal and vertical termini of glaciers need to be represented, or it can
be used on its own, if the glaciers terminate and calve near the grounding
line, i.e., if they have no significant ice shelf cavity beneath them.

Pkg/icefront is separated from pkg/shelfice because

(i) rather than search 2D space for ice shelf cavities, the icefront package
needs to search a 3D space for the ice fronts,

(ii) probably best to save position of the ice fronts in vectors during
initialization in the same way as is done in pkg/downslope,

(iii) each grid box can have up to three separate contributions from ice
fronts while ice shelf cavities only have one contribution from the top,

(iv) shelfice_thermodynamics.F can be rewritten using the downslope
formulation (and also shelfice_u_drag.F and shelfice_v_drag.F I think) but
they would be radically different and harder to read than they are right now,

(v) eventually the physics of the two packages may diverge to include
different processes, for example, a fancier boundary layer scheme for shelfice
and a calving scheme for icefront, and

(vi) it's less coding effort to separate the two packages.

Pkg/icefront is mostly designed to represent Greenland glaciers in model
configurations that can resolve the Greenland fjords and glacier outlets,
i.e., that have horizontal grid spacing of O(1km) or less.
